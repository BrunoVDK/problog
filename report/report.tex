\documentclass{res/theme}
\usepackage{wallpaper}
\usepackage{tabu,longtable,booktabs,array}
\def\changemargin#1#2{\list{}{\rightmargin#2\leftmargin#1}\item[]}
\let\endchangemargin=\endlist 
\newcommand{\block}[1]{\vspace{1.4cm}\begin{changemargin}{2.5cm}{2.5cm}\large\textit{#1}\end{changemargin}\vspace{0.6cm}}
\ThisCenterWallPaper{0.7}{res/cover.png}
\usepackage{microtype} % Better word wrapping
  \usepackage{color}
\usepackage{xcolor}
\usepackage{titlesec}
\titleformat{\section}
  {\normalfont\Large}
  {\thesection}{1em}{}
\titleformat{\subsection}
  {\normalfont\large}
  {\thesection}{1em}{}
\usepackage{hyperref}
\usepackage{multicol}
\definecolor{blue}{rgb}{0.01, 0.28, 1.0}
\definecolor{lightgray}{rgb}{0.95, 0.95, 0.95}
\hypersetup{colorlinks=true, urlcolor=blue, citecolor=blue, filecolor=blue, linkcolor=blue}
\usepackage{afterpage}
\pagecolor{lightgray}\afterpage{\nopagecolor}
\usepackage[firstpage=true]{background}
\def\Step{0.1cm}
\tikzset{dotted lines/.style={black, loosely dotted,  thick}}
\backgroundsetup{
scale=1,
angle=0,
position={0,0},
contents={
\begin{tikzpicture}[remember picture,overlay]
\draw[dotted lines,step=\Step,help lines]
  (-10,-30) grid  (20,10);
  \end{tikzpicture}
  }
}
\usepackage[font={footnotesize}]{caption}
\usepackage[ruled,vlined]{algorithm2e}
\newenvironment{snippet}[1][htb]
  {\renewcommand{\algorithmcfname}{Code Snippet}% Update algorithm name
   \vspace{0.3cm}\begin{algorithm}[#1]%
  }{\end{algorithm}\vspace{0.3cm}}
\usepackage{array}
\usepackage{tabu}
\usepackage{subfig}
\usepackage{multirow}
\usepackage{float}
\usepackage[all]{hypcap}
\usepackage{etoolbox}
\usepackage{xfrac}
\usepackage{wrapfig}
\usepackage{lscape}
\usepackage{rotating}
\usepackage{epstopdf}
\usepackage{fancyvrb}
\usepackage{tikz}
\usetikzlibrary{bayesnet}
\usetikzlibrary{arrows}
\usetikzlibrary{backgrounds}
\usepackage{amssymb}
\makeatletter
\patchcmd{\@BVerbatim}
  {\BVerbatim@font}
  {\BVerbatim@font\footnotesize}
  {}{}
\makeatother

%%%%%%%%%%%%%%%%%%%%%%%%%%%%%%%%%%%
%%%%%%%%%%%%%%%%%%%%%%%%%%%%%%%%%%%
%%%%%%%%%%%%%%%%%%%%%%%%%%%%%%%%%%%

%%% Authors
\addAuthor{Michiel}{Janssen}{}
\addAuthor{Bruno}{Vandekerkhove}{}

%%% Title page
\casename{Probabilistic Programming}
\subtitle{Michiel Janssen \& Bruno Vandekerkhove}
\course{{H05N0A: Capita Selecta : Artificial Intelligence}}
\academicyear{2020}
\newcommand{\tbd}{\textbf{To be discussed}}

%%%
%%% Start Document
%%%
\begin{document}
\maketitle
\tableofcontents

\vspace{0.2cm}
\begin{center}
\begin{tikzpicture}
\draw [dotted] (0,1) -- (15,1);
\end{tikzpicture}
\end{center}

%
% Introduction
%
\begin{center}
\textit{Below's our solution for the given challenges. The questions in each section of the original assignment are answered in a section having the same title.}
\end{center}

\vspace{0.0cm}
\begin{code}
\begin{minted}[xleftmargin=20pt,linenos]{PROLOG}
person(a). 
person(b). 
person(c). 
0.2::stress(X) :- person(X). 
0.1::friends(X,Y) :- person(X), person(Y). 
0.3::smokes(X) :- stress(X). 
0.4::smokes(X) :- friends(X,Y), smokes(Y). 
query(smokes(a)).
\end{minted}
\captionof{listing}{\texttt{PROBLOG} program used throughout the first two chapters of the report.}
\label{code:base}
\end{code}

\begin{center}
\begin{tikzpicture}
\draw [dotted] (0,1) -- (15,1);
\end{tikzpicture}
\end{center}

\begin{multicols*}{2}
\raggedcolumns

%
% Probabilistic Inference
% 
\fakesection{Probabilistic Inference Using Weighted Model Counting}

%%%
%%%
%%%

\fakesubsection{SRL to CNF}

First the program is grounded. This is a matter of collecting all atoms involved in all proofs of the query.

\begin{code}
\begin{minted}[xleftmargin=20pt,linenos]{PROLOG}
0.2::stress(a).
0.2::stress(b).
0.2::stress(c).

0.1::friends(a,a).
0.1::friends(a,b).
0.1::friends(a,c).

0.1::friends(b,a).
0.1::friends(b,b).
0.1::friends(b,c).

0.1::friends(c,a).
0.1::friends(c,b).
0.1::friends(c,c).

0.3::smokes(a) :- stress(a).
0.3::smokes(b) :- stress(b).
0.3::smokes(c) :- stress(c).

0.4::smokes(a) :- friends(a,a), smokes(a).
0.4::smokes(a) :- friends(a,b), smokes(b).
0.4::smokes(a) :- friends(a,c), smokes(c).
0.4::smokes(b) :- friends(b,a), smokes(a).
0.4::smokes(b) :- friends(b,b), smokes(b).
0.4::smokes(b) :- friends(b,c), smokes(c).

0.4::smokes(c) :- friends(c,a), smokes(a).
0.4::smokes(c) :- friends(c,b), smokes(b).
0.4::smokes(c) :- friends(c,c), smokes(c).
\end{minted}
\captionof{listing}{Relevant ground program.}
\label{code:base}
\vspace{0.5cm}
\end{code}

\definecolor{darkgray}{rgb}{0.4,0.4,0.4}
\definecolor{c1}{rgb}{0.83,0.13,0.18}
\definecolor{c2}{rgb}{0.23,0.48,0.34}
\definecolor{c3}{rgb}{0.18,0.35,0.58}
\begin{figure*}[h]
\centering
\begin{tikzpicture}[scale=0.85]

	\node at (0,0) (1) {\texttt{\textcolor{c1}{smokes(a)}}};
	
	\node at (0,-2) (2) {\texttt{stress(a)}};
	\node at (0,-4) (6) {\texttt{stress(b)}};
	\node at (0,-6) (7) {\texttt{stress(c)}};
	
	\node at (4,-2) (3) {\texttt{friends(a,a),\textcolor{c1}{smokes(a)}}};
	\node at (8,-3) (4) {\texttt{friends(a,b),\textcolor{c2}{smokes(b)}}};
	\node at (12,-4) (5) {\texttt{friends(a,c),\textcolor{c3}{smokes(c)}}};
	
	\node at (4,-5) (8) {\texttt{friends(b,a),\textcolor{c1}{smokes(a)}}};
	\node at (4,-6) (9) {\texttt{friends(b,b),\textcolor{c2}{smokes(b)}}};
	\node at (4,-7) (10) {\texttt{friends(b,c),\textcolor{c3}{smokes(c)}}};
	
	\node at (7,-8) (11) {\texttt{friends(c,a),\textcolor{c1}{smokes(a)}}};
	\node at (7,-9) (12) {\texttt{friends(c,b),\textcolor{c2}{smokes(b)}}};
	\node at (7,-10) (13) {\texttt{friends(c,c),\textcolor{c3}{smokes(c)}}};
	
	\draw[darkgray,->] (1.south) to (2.north);
	
	\draw[darkgray,->] (1.east) to [out=0,in=90,looseness=0.5](3.north);
	\draw[darkgray,->] (1.east) to [out=0,in=90,looseness=0.5](4.north);
	\draw[darkgray,->] (1.east) to [out=0,in=90,looseness=0.5](5.north);
	
	\draw[darkgray,->] (4.west) to [out=180,in=90,looseness=0.5](6.north);
	\draw[darkgray,->] (5.west) to [out=180,in=90,looseness=0.5](7.north);
	
	\draw[darkgray,->] ([xshift=6mm]4.south) to [out=-90,in=0,looseness=0.5](8.east);
	\draw[darkgray,->] ([xshift=6mm]4.south) to [out=-90,in=0,looseness=0.5](9.east);
	\draw[darkgray,->] ([xshift=6mm]4.south) to [out=-90,in=0,looseness=0.5](10.east);
	
	\draw[darkgray,->] ([xshift=6mm]5.south) to [out=-90,in=0,looseness=0.5](11.east);
	\draw[darkgray,->] ([xshift=6mm]5.south) to [out=-90,in=0,looseness=0.5](12.east);
	\draw[darkgray,->] ([xshift=6mm]5.south) to [out=-90,in=0,looseness=0.5](13.east);
	
	% Loops
	
	%\draw[->,dashed,blue] ([xshift=-4mm]3.north) to ([xshift=4mm]1.south);
%	
%	\draw[->,dashed,blue] ([xshift=-4mm]8.north) to ([xshift=4mm,looseness=0.5]1.south);
%	\draw[->,dashed,blue] ([xshift=-4mm]9.east) to ([xshift=4mm,looseness=0.5]4.south);
%	\draw[->,dashed,blue] ([xshift=-4mm]10.east) to ([xshift=4mm,looseness=0.5]5.south);
%	
%	\draw[->,dashed,blue] ([xshift=-4mm]11.east) to ([xshift=4mm,looseness=0.5]1.south);
%	\draw[->,dashed,blue] ([xshift=-4mm]12.east) to ([xshift=4mm,looseness=0.5]4.south);
%	\draw[->,dashed,blue] ([xshift=-4mm]13.east) to ([xshift=4mm,looseness=0.5]5.south);
	
%	\node at (-2.5,5) (3) {$\mathbb{N}$};
%	\draw[->,shorten >=1pt] (3) to [out=180,in=-90,loop,looseness=15] node[left]{$id$} (3);
%	\draw[->,shorten >=1pt] (3) to [out=180,in=-90,loop,looseness=30] node[left]{$add1mod3$} (3);
%	\draw[->,shorten >=1pt] (3) to [out=180,in=-90,loop,looseness=45] node[left]{$add2mod3$} (3);
\end{tikzpicture}
\caption{SLG-tree produced while turning the ground program into a boolean formula. Coloured atoms indicate the presence of cycles.}
\label{fig:nestedtries}
\end{figure*}

\noindent The proofs of the query make for a trie as shown in figure \ref{fig:nestedtries}, where colourings indicate the presence of cycles. Any proof involving an atom \texttt{friends(X,X)} or \texttt{friends(Y,a)} (with $Y\in\{b,c\}$) is non-minimal and doesn't affect the final probability. These atoms are disregarded. For the remaining cycles (involving \texttt{friends(b,c)} and \texttt{friends(c,b)}) auxiliary variables can be used to obtain a cycle-free program :

\begin{code}
\begin{minted}[xleftmargin=20pt,linenos]{PROLOG}
0.2::stress(a).
0.2::stress(b).
0.2::stress(c).

0.1::friends(a,b).
0.1::friends(a,c).
0.1::friends(b,c).
0.1::friends(c,b).

0.3::p(a).
0.3::p(b).
0.3::p(c).

0.4::p(a,b).
0.4::p(a,c).
0.4::p(b,c).
0.4::p(c,b).

smokes(a) :- stress(a), p(a).
smokes(b) :- stress(b), p(b).
smokes(c) :- stress(c), p(c).

smokes(a) :- 
    friends(a,b), smokes(b), p(a,b).
smokes(a) :-
    friends(a,c), smokes(c), p(a,c).
smokes(b) :- 
    friends(b,c), stress(c), p(c), p(b,c).
smokes(c) :- 
    friends(c,b), stress(b), p(b), p(c,b).

query(smokes(a)).
\end{minted}
\captionof{listing}{Relevant ground program without cycles.}
\label{code:base}
\vspace{0.5cm}
\end{code}

\noindent The above logic program is equivalent to the following propositional formula :
\begin{center}
\begin{tabular}{ll}
$smokes(a)\leftrightarrow$ & $stress(a)$\\
& $\lor\ (friends(a,b) \land smokes(b))$\\
& $\lor\ (friends(a,c) \land smokes(c))$\\
$\land\ (smokes(b)\leftrightarrow$ & $p(b) \lor (friends(b,c) \land p(c)))$ \\
$\land\ (smokes(c)\leftrightarrow$ & $p(c) \lor (friends(c,b) \land p(b)))$ \\
$\land\ (p(b)\leftrightarrow$ & $stress(b))$ \\
$\land\ (p(c)\leftrightarrow$ & $stress(c))$ \\
\end{tabular}
\end{center}

\par\noindent Which gives the following \texttt{CNF} :\vspace{0.3cm}\\
$(\lnot smokes(a) \lor stress(a) \lor friends(a,b) \lor friends(a,c))
\\\land (\lnot smokes(a) \lor stress(a) \lor friends(a,b) \lor smokes(c))
\\\land (\lnot smokes(a) \lor stress(a) \lor smokes(b) \lor friends(a,c))
\\\land (\lnot smokes(a) \lor stress(a) \lor smokes(b) \lor smokes(c))
\\\land (\lnot stress(a) \lor smokes(a))
\\\land (\lnot friends(a,b) \lor \lnot smokes(b) \lor smokes(a))
\\\land (\lnot friends(a,c) \lor \lnot smokes(c) \lor smokes(a))
\\\land (\lnot smokes(b) \lor p(b) \lor friends(b,c))
\\\land (\lnot smokes(b) \lor p(b) \lor p(c))
\\\land (\lnot p(b) \lor smokes(b))
\\\land (\lnot friends(b,c) \lor \lnot p(c) \lor smokes(b))
\\\land (\lnot smokes(c) \lor p(c) \lor friends(c,b))
\\\land (\lnot smokes(c) \lor p(c) \lor p(b))
\\\land (\lnot p(c) \lor smokes(c))
\\\land (\lnot friends(c,b) \lor \lnot p(b) \lor smokes(c))
\\\land (\lnot p(b) \lor stress(b)) 
\\\land (\lnot stress(b) \lor p(b)) 
\\\land (\lnot p(c) \lor stress(c)) 
\\\land (\lnot stress(c) \lor p(c))$
\vspace{0.3cm}

\par\noindent The probabilistic literals \texttt{CNF} are assigned weights (derived literals have a weight of 1) :

\begin{center}
\begin{tabular}{c|c}
Atom & Weight \\
\hline
stress(a) & 0.2 \\
$\lnot$stress(a) & 0.8 \\
stress(b) & 0.2 \\
$\lnot$stress(b) & 0.8 \\
stress(c) & 0.2 \\
$\lnot$stress(c) & 0.8 \\
friends(a,b) & 0.1 \\
$\lnot$friends(a,b) & 0.9 \\
friends(a,c) & 0.1 \\
$\lnot$friends(a,c) & 0.9 \\
friends(b,c) & 0.1 \\
$\lnot$friends(b,c) & 0.9 \\
friends(c,b) & 0.1 \\
$\lnot$friends(c,b) & 0.9 \\
\end{tabular}
\end{center}

%%%
%%%
%%%

\fakesubsection{SRL to PGM}



%%%
%%%
%%%

\fakesubsection{PGM to CNF}



%%%
%%%
%%%

\fakesubsection{Weighted Model Counting}

%
% Lifted Inference
% 
\fakesection{Lifted Inference}

%%%
%%%
%%%

\fakesubsection{Calculating Probability with Probabilistic Databases}

\noindent Starting from the following probabilistic database tables :

\begin{table}[h]

	\centering
	
	\begin{subtable}{.2\linewidth}{
	\centering
	\begin{tabular}{c|c}
		$X$ & $\underline{stress(X)}$\\
		$a$ & 0.2\\
		$b$ & 0.2\\
		$c$ & 0.2
	\end{tabular}}
	\label{tab:1a}
	\end{subtable}
	\begin{subtable}{.3\linewidth}{
	\centering
	\begin{tabular}{cc|c}
		$X$ & $Y$ & $\underline{friends(X,Y)}$\\
		$a$ & $b$ & 0.1\\
		$a$ & $c$ & 0.1\\
		$b$ & $c$ & 0.1\\
		$c$ & $b$ & 0.1\\
	\end{tabular}}
	\label{tab:2a}
	\end{subtable}
	\begin{subtable}{.2\linewidth}{
	\centering
	\begin{tabular}{c|c}
		$X$ & $\underline{p(X)}$\\
		$a$ & 0.3\\
		$b$ & 0.3\\
		$c$ & 0.3\\
	\end{tabular}}
	\label{tab:3a}
	\end{subtable}
	\begin{subtable}{.2\linewidth}{
	\centering
	\begin{tabular}{cc|c}
		$X$ & $Y$ & $\underline{p(X,Y)}$\\
		$a$ & $b$ & 0.4\\
		$a$ & $c$ & 0.4\\
		$b$ & $c$ & 0.4\\
		$c$ & $b$ & 0.4\\
	\end{tabular}}
	\label{tab:4a}
	\end{subtable}

	\caption{Probabilistic database to be used for querying.}\label{tab:probdata}

\end{table}

\noindent Querying for $smokes(a)$ can be done as follows :
$$(stress(a)\ \land\ p(a))\ \lor\ (\exists x: p(a,x)\ \land\ friends(a,x)\ \land\ ((stress(x)\ \land\ p(x))\ \lor\ \exists y: p(x,y)\ \land\ friends(x,y)\ \land\ stress(y)\ \land\ p(y)))$$
Rules can be applied to this formula. Applying the inclusive or rule gives :
\begin{gather*}
1-(1-p_{stress(a)\ \land\ p(a)})\times (1-p_{s_1})
\end{gather*}
Where $s_1$ refers to the subformula for the first existential operator. The probability $p_{s_1}$ cannot simply be calculated by exponentiation as the various instantiations of $s_1$ are dependent. However, there are only two such instantiations (for $smokes(a)$ and $smokes(b)$) and since the existential operator generalises the logical or, the formula for exclusive or can directly be used instead :
\begin{gather*}
p_{s_1} = p_{friends(a,b)}\times p_{p(a,b)}\times p_{smokes(b)} + p_{friends(a,c)}\times p_{p(a,c)}\times p_{smokes(c)} - p_{both}
\end{gather*}
Where $p_{both}=p_{friends(a,b)}\ \land\ p_{p(a,b)}\ \land\ p_{friends(a,c)}\ \land\ p_{p(a,c)}\ \land\ p_{smokes(b)}\ \land\ p_{smokes(c)}$. The probability of either $smokes(b)$ or $smokes(c)$ is the same (as is the probability of $friends(a,b)\ \land\ p(a,b)$ and $friends(a,c)\ \land\ p(a,c)$, so this reduces to :
\begin{gather*}
p_{s_1} = 2\times p_{friends(a,b)}\times p_{p(a,b)}\times p_{smokes(b)} - p_{friends(a,b)\ \land\  friends(a,c)\ \land\ smokes(b)\ \land\ smokes(c)}
\end{gather*}
The probability of $smokes(b)$ can be calculated using the same rules (an inclusive or and the and-rule) :
\begin{gather*}
1-(1-p_{stress(b)}\times p_{p(b)})\times (1-p_{friends(b,c)}\times p_{p(b,c)}\times p_{stress(c)}\times p_{p(c)})\\
= (1-(1-0,2\times 0,3)\times (1-0,1\times 0,4\times 0,3\times 0,2)\\
\approx 0.62256
\end{gather*}
The probability of $p_{smokes(b)\ \land\ smokes(c)}$ can also be calculated with these rules, for example :
\begin{gather*}
p_{smokes(b)\ \land\ smokes(c)} \\
= p_{stress(b)\ \land\ p(b)\ \land\ smokes(c)} + p_{stress(c)\ \land\ p(c)\ \land\ smokes(b)} - p_{all}\\
= 2\times p_{stress(b)}\times p_{p(b)}\times (p_{stress(c)\ \land\ p(c)}+p_{friends(b,c)\ \land\ p(b,c)}-p_{stress(c)\ \land\ p(c)\ \land\ friends(b,c)\ \land\ p(b,c)})\\
- p_{stress(b)}\times p_{p(b)}\times p_{stress(c)}\times p_{p(c)}
\end{gather*}
The probability of $p_{stress(a)\ \land\ p(a)}$ equals $p_{p(a)}\times p_{stress(a)}$.
After filling in all the missing probabilities of the involved literals (by looking them up in the database) the following total probability is obtained :
% 1-(1-0,2*0,3)*(1-(2*0,1*0,4*(1-(1-0,2*0,3)*(1-0,1*0,4*0,3*0,2))-0,1*0,4*0,4*0,1*(0,2*0,3*(0,2*0,3+0,1*0,4-0,2*0,3*0,1*0,4)+0,2*0,3*(0,2*0,3+0,1*0,4-0,2*0,3*0,1*0,4)-0,2*0,3*0,2*0,3)))
\begin{gather*}
1-(1-0,2\times 0,3)\times (1-(2\times 0,1\times 0,4\times (1-(1-0,2\times 0,3)\times (1-0,1\times 0,4\times 0,3\times 0,2))-\\
0,1\times 0,4\times 0,1\times 0,4\times (0,2\times 0,3\times (0,2\times 0,3+0,1\times 0,4-0,2\times 0,3\times 0,1\times 0,4)+0,2\times 0,3\times\\ 
(0,2\times 0,3+0,1\times 0,4-0,2\times 0,3\times 0,1\times 0,4)-0,2\times 0,3\times 0,2\times 0,3)))\\
\approx 0,06466945075
\end{gather*}This is the same number as the one previously found by the weighted model counters. Some symmetry was taken advantage of.

\fakesubsection{Skolemization \& Noisy OR}

A new encoding can be made by, instead of converting the formula $X\Leftrightarrow Y_1\lor Y_2 \lor ... \lor Y_n$ to CNF in the usual way, applying a Tseitin transformation. This leads to :
\begin{gather*}
X \lor \lnot Y_1\\
X \lor \lnot Y_2\\
...\\
X \lor \lnot Y_n\\
X \lor T\\
T \lor \lnot Y_1\\
T \lor \lnot Y_2\\
...\\
T \lor \lnot Y_n
\end{gather*}
Applying this transformation to the noisy OR relations in the encoding lead to a slightly \textit{larger} circuit though, when tested with a program in which there were more ancestors, the resulting circuits were \textit{smaller}.

%
% Parameter Learning
% 
\fakesection{Parameter Learning}

%%%
%%%
%%%

Learning from interpretations can be done by compiling these interpretations together with the base program into some kind of structure such that inference becomes tractable. In the algorithm that was written, a d-DNNF is generated for each interpretation (an SDD in particular since SharpSAT had trouble with a known issue). Then, the weights of the parameters of interest are updated iteratively until convergence i.e. until new weights don't differ much from the previous ones. The resulting (local) optimum approximates the parameter's real values.\\

\par\noindent The d-DNNFs are generated from CNFs with \texttt{PySDD}. During each iteration they are used to calculate marginals based on current values of the parameters. Their structure never changes, only their weights are updated as the algorithm progresses. Either because the new estimates have to be taken into account at the end of every iteration, or because the marginals have to be calculated (which is done by toying with the weights rather than by applying Bayes' theorem which was the approach used in section 1).\\

\par\noindent The algorithm is reasonably fast and some results are shown below :

\begin{table}[h]
\centering
\begin{tabular}{ccc}
& &\\\hline
\end{tabular}
\caption{Results of the parameter learning algorithm. Initial weights were always set to 0.5 for the sake of reproducibility (setting them randomly is a matter of commenting out a line). On a regular computer running for 1000 iterations on 1000 examples took no longer than a minute. Various tests were done such as comparing with ProbLog's own system, using the same initial values. }
\label{plres}
\end{table}

\end{multicols*}

%
% Appendix
%

\newpage
\appendix
\fakesection{Appendix}

%%%
%%%
%%%

\fakesubsection{Bayesian Network Encodings}

\fakesubsubsection{Indicator Clauses}

$
\lambda_{stress(a)=true} \lor \lambda_{stress(a)=false}\\
\lnot\lambda_{stress(a)=true} \lor \lnot\lambda_{stress(a)=false}\\
\lambda_{stress(b)=true} \lor \lambda_{stress(b)=false}\\
\lnot\lambda_{stress(b)=true} \lor \lnot\lambda_{stress(b)=false}\\
\lambda_{stress(c)=true} \lor \lambda_{stress(c)=false}\\
\lnot\lambda_{stress(c)=true} \lor \lnot\lambda_{stress(c)=false}\\
\lambda_{aux1=true} \lor \lambda_{aux1=false}\\
\lnot\lambda_{aux1=true} \lor \lnot\lambda_{aux1=false}\\
\lambda_{aux2=true} \lor \lambda_{aux2=false}\\
\lnot\lambda_{aux2=true} \lor \lnot\lambda_{aux2=false}\\
\lambda_{aux3=true} \lor \lambda_{aux3=false}\\
\lnot\lambda_{aux3=true} \lor \lnot\lambda_{aux3=false}\\
\lambda_{friends(a,b)=true} \lor \lambda_{friends(a,b)=false}\\
\lnot\lambda_{friends(a,b)=true} \lor \lnot\lambda_{friends(a,b)=false}\\
\lambda_{friends(a,c)=true} \lor \lambda_{friends(a,c)=false}\\
\lnot\lambda_{friends(a,c)=true} \lor \lnot\lambda_{friends(a,c)=false}\\
\lambda_{friends(b,c)=true} \lor \lambda_{friends(b,c)=false}\\
\lnot\lambda_{friends(b,c)=true} \lor \lnot\lambda_{friends(b,c)=false}\\
\lambda_{friends(c,b)=true} \lor \lambda_{friends(c,b)=false}\\
\lnot\lambda_{friends(c,b)=true} \lor \lnot\lambda_{friends(c,b)=false}\\
\lambda_{aux4=true} \lor \lambda_{aux4=false}\\
\lnot\lambda_{aux4=true} \lor \lnot\lambda_{aux4=false}\\
\lambda_{aux5=true} \lor \lambda_{aux5=false}\\
\lnot\lambda_{aux5=true} \lor \lnot\lambda_{aux5=false}\\
\lambda_{smokes(b)=true} \lor \lambda_{smokes(b)=false}\\
\lnot\lambda_{smokes(b)=true} \lor \lnot\lambda_{smokes(b)=false}\\
\lambda_{smokes(c)=true} \lor \lambda_{smokes(c)=false}\\
\lnot\lambda_{smokes(c)=true} \lor \lnot\lambda_{smokes(c)=false}\\
\lambda_{aux6=true} \lor \lambda_{aux6=false}\\
\lnot\lambda_{aux6=true} \lor \lnot\lambda_{aux6=false}\\
\lambda_{aux7=true} \lor \lambda_{aux7=false}\\
\lnot\lambda_{aux7=true} \lor \lnot\lambda_{aux7=false}\\
\lambda_{smokes(a)=true} \lor \lambda_{smokes(a)=false}\\
\lnot\lambda_{smokes(a)=true} \lor \lnot\lambda_{smokes(a)=false}
$

\fakesubsubsection{ENC1 (Parameter Clauses)}
$
\lambda_{stress(a)=true} \Leftrightarrow \theta_{stress(a)=true}\\
\lambda_{stress(a)=false} \Leftrightarrow \theta_{stress(a)=false}\\
\lambda_{stress(b)=true} \Leftrightarrow \theta_{stress(b)=true}\\
\lambda_{stress(b)=false} \Leftrightarrow \theta_{stress(b)=false}\\
\lambda_{stress(c)=true} \Leftrightarrow \theta_{stress(c)=true}\\
\lambda_{stress(c)=false} \Leftrightarrow \theta_{stress(c)=false}\\
\lambda_{stress(a)=true} \land \lambda_{aux1=true} \Leftrightarrow \theta_{aux1=true|stress(a)=true}\\
\lambda_{stress(a)=true} \land \lambda_{aux1=false} \Leftrightarrow \theta_{aux1=false|stress(a)=true}\\
\lambda_{stress(a)=false} \land \lambda_{aux1=true} \Leftrightarrow \theta_{aux1=true|stress(a)=false}\\
\lambda_{stress(a)=false} \land \lambda_{aux1=false} \Leftrightarrow \theta_{aux1=false|stress(a)=false}\\
\lambda_{stress(b)=true} \land \lambda_{aux2=true} \Leftrightarrow \theta_{aux2=true|stress(b)=true}\\
\lambda_{stress(b)=true} \land \lambda_{aux2=false} \Leftrightarrow \theta_{aux2=false|stress(b)=true}\\
\lambda_{stress(b)=false} \land \lambda_{aux2=true} \Leftrightarrow \theta_{aux2=true|stress(b)=false}\\
\lambda_{stress(b)=false} \land \lambda_{aux2=false} \Leftrightarrow \theta_{aux2=false|stress(b)=false}\\
\lambda_{stress(c)=true} \land \lambda_{aux3=true} \Leftrightarrow \theta_{aux3=true|stress(c)=true}\\
\lambda_{stress(c)=true} \land \lambda_{aux3=false} \Leftrightarrow \theta_{aux3=false|stress(c)=true}\\
\lambda_{stress(c)=false} \land \lambda_{aux3=true} \Leftrightarrow \theta_{aux3=true|stress(c)=false}\\
\lambda_{stress(c)=false} \land \lambda_{aux3=false} \Leftrightarrow \theta_{aux3=false|stress(c)=false}\\
\lambda_{friends(a,b)=true} \Leftrightarrow \theta_{friends(a,b)=true}\\
\lambda_{friends(a,b)=false} \Leftrightarrow \theta_{friends(a,b)=false}\\
\lambda_{friends(a,c)=true} \Leftrightarrow \theta_{friends(a,c)=true}\\
\lambda_{friends(a,c)=false} \Leftrightarrow \theta_{friends(a,c)=false}\\
\lambda_{friends(b,c)=true} \Leftrightarrow \theta_{friends(b,c)=true}\\
\lambda_{friends(b,c)=false} \Leftrightarrow \theta_{friends(b,c)=false}\\
\lambda_{friends(c,b)=true} \Leftrightarrow \theta_{friends(c,b)=true}\\
\lambda_{friends(c,b)=false} \Leftrightarrow \theta_{friends(c,b)=false}\\
\lambda_{friends(b,c)=true} \land \lambda_{aux3=true} \land \lambda_{aux4=true} \Leftrightarrow \theta_{aux4=true|friends(b,c)=true \land aux3=true}\\
\lambda_{friends(b,c)=true} \land \lambda_{aux3=true} \land \lambda_{aux4=false} \Leftrightarrow \theta_{aux4=false|friends(b,c)=true \land aux3=true}\\
\lambda_{friends(b,c)=true} \land \lambda_{aux3=false} \land \lambda_{aux4=true} \Leftrightarrow \theta_{aux4=true|friends(b,c)=true \land aux3=false}\\
\lambda_{friends(b,c)=true} \land \lambda_{aux3=false} \land \lambda_{aux4=false} \Leftrightarrow \theta_{aux4=false|friends(b,c)=true \land aux3=false}\\
\lambda_{friends(b,c)=false} \land \lambda_{aux3=true} \land \lambda_{aux4=true} \Leftrightarrow \theta_{aux4=true|friends(b,c)=false \land aux3=true}\\
\lambda_{friends(b,c)=false} \land \lambda_{aux3=true} \land \lambda_{aux4=false} \Leftrightarrow \theta_{aux4=false|friends(b,c)=false \land aux3=true}\\
\lambda_{friends(b,c)=false} \land \lambda_{aux3=false} \land \lambda_{aux4=true} \Leftrightarrow \theta_{aux4=true|friends(b,c)=false \land aux3=false}\\
\lambda_{friends(b,c)=false} \land \lambda_{aux3=false} \land \lambda_{aux4=false} \Leftrightarrow \theta_{aux4=false|friends(b,c)=false \land aux3=false}\\
\lambda_{friends(c,b)=true} \land \lambda_{aux2=true} \land \lambda_{aux5=true} \Leftrightarrow \theta_{aux5=true|friends(c,b)=true \land aux2=true}\\
\lambda_{friends(c,b)=true} \land \lambda_{aux2=true} \land \lambda_{aux5=false} \Leftrightarrow \theta_{aux5=false|friends(c,b)=true \land aux2=true}\\
\lambda_{friends(c,b)=true} \land \lambda_{aux2=false} \land \lambda_{aux5=true} \Leftrightarrow \theta_{aux5=true|friends(c,b)=true \land aux2=false}\\
\lambda_{friends(c,b)=true} \land \lambda_{aux2=false} \land \lambda_{aux5=false} \Leftrightarrow \theta_{aux5=false|friends(c,b)=true \land aux2=false}\\
\lambda_{friends(c,b)=false} \land \lambda_{aux2=true} \land \lambda_{aux5=true} \Leftrightarrow \theta_{aux5=true|friends(c,b)=false \land aux2=true}\\
\lambda_{friends(c,b)=false} \land \lambda_{aux2=true} \land \lambda_{aux5=false} \Leftrightarrow \theta_{aux5=false|friends(c,b)=false \land aux2=true}\\
\lambda_{friends(c,b)=false} \land \lambda_{aux2=false} \land \lambda_{aux5=true} \Leftrightarrow \theta_{aux5=true|friends(c,b)=false \land aux2=false}\\
\lambda_{friends(c,b)=false} \land \lambda_{aux2=false} \land \lambda_{aux5=false} \Leftrightarrow \theta_{aux5=false|friends(c,b)=false \land aux2=false}\\
\lambda_{aux4=true} \land \lambda_{aux2=true} \land \lambda_{smokes(b)=true} \Leftrightarrow \theta_{smokes(b)=true|aux4=true \land aux2=true}\\
\lambda_{aux4=true} \land \lambda_{aux2=true} \land \lambda_{smokes(b)=false} \Leftrightarrow \theta_{smokes(b)=false|aux4=true \land aux2=true}\\
\lambda_{aux4=true} \land \lambda_{aux2=false} \land \lambda_{smokes(b)=true} \Leftrightarrow \theta_{smokes(b)=true|aux4=true \land aux2=false}\\
\lambda_{aux4=true} \land \lambda_{aux2=false} \land \lambda_{smokes(b)=false} \Leftrightarrow \theta_{smokes(b)=false|aux4=true \land aux2=false}\\
\lambda_{aux4=false} \land \lambda_{aux2=true} \land \lambda_{smokes(b)=true} \Leftrightarrow \theta_{smokes(b)=true|aux4=false \land aux2=true}\\
\lambda_{aux4=false} \land \lambda_{aux2=true} \land \lambda_{smokes(b)=false} \Leftrightarrow \theta_{smokes(b)=false|aux4=false \land aux2=true}\\
\lambda_{aux4=false} \land \lambda_{aux2=false} \land \lambda_{smokes(b)=true} \Leftrightarrow \theta_{smokes(b)=true|aux4=false \land aux2=false}\\
\lambda_{aux4=false} \land \lambda_{aux2=false} \land \lambda_{smokes(b)=false} \Leftrightarrow \theta_{smokes(b)=false|aux4=false \land aux2=false}\\
\lambda_{aux5=true} \land \lambda_{aux3=true} \land \lambda_{smokes(c)=true} \Leftrightarrow \theta_{smokes(c)=true|aux5=true \land aux3=true}\\
\lambda_{aux5=true} \land \lambda_{aux3=true} \land \lambda_{smokes(c)=false} \Leftrightarrow \theta_{smokes(c)=false|aux5=true \land aux3=true}\\
\lambda_{aux5=true} \land \lambda_{aux3=false} \land \lambda_{smokes(c)=true} \Leftrightarrow \theta_{smokes(c)=true|aux5=true \land aux3=false}\\
\lambda_{aux5=true} \land \lambda_{aux3=false} \land \lambda_{smokes(c)=false} \Leftrightarrow \theta_{smokes(c)=false|aux5=true \land aux3=false}\\
\lambda_{aux5=false} \land \lambda_{aux3=true} \land \lambda_{smokes(c)=true} \Leftrightarrow \theta_{smokes(c)=true|aux5=false \land aux3=true}\\
\lambda_{aux5=false} \land \lambda_{aux3=true} \land \lambda_{smokes(c)=false} \Leftrightarrow \theta_{smokes(c)=false|aux5=false \land aux3=true}\\
\lambda_{aux5=false} \land \lambda_{aux3=false} \land \lambda_{smokes(c)=true} \Leftrightarrow \theta_{smokes(c)=true|aux5=false \land aux3=false}\\
\lambda_{aux5=false} \land \lambda_{aux3=false} \land \lambda_{smokes(c)=false} \Leftrightarrow \theta_{smokes(c)=false|aux5=false \land aux3=false}\\
\lambda_{friends(a,b)=true} \land \lambda_{smokes(b)=true} \land \lambda_{aux6=true} \Leftrightarrow \theta_{aux6=true|friends(a,b)=true \land smokes(b)=true}\\
\lambda_{friends(a,b)=true} \land \lambda_{smokes(b)=true} \land \lambda_{aux6=false} \Leftrightarrow \theta_{aux6=false|friends(a,b)=true \land smokes(b)=true}\\
\lambda_{friends(a,b)=true} \land \lambda_{smokes(b)=false} \land \lambda_{aux6=true} \Leftrightarrow \theta_{aux6=true|friends(a,b)=true \land smokes(b)=false}\\
\lambda_{friends(a,b)=true} \land \lambda_{smokes(b)=false} \land \lambda_{aux6=false} \Leftrightarrow \theta_{aux6=false|friends(a,b)=true \land smokes(b)=false}\\
\lambda_{friends(a,b)=false} \land \lambda_{smokes(b)=true} \land \lambda_{aux6=true} \Leftrightarrow \theta_{aux6=true|friends(a,b)=false \land smokes(b)=true}\\
\lambda_{friends(a,b)=false} \land \lambda_{smokes(b)=true} \land \lambda_{aux6=false} \Leftrightarrow \theta_{aux6=false|friends(a,b)=false \land smokes(b)=true}\\
\lambda_{friends(a,b)=false} \land \lambda_{smokes(b)=false} \land \lambda_{aux6=true} \Leftrightarrow \theta_{aux6=true|friends(a,b)=false \land smokes(b)=false}\\
\lambda_{friends(a,b)=false} \land \lambda_{smokes(b)=false} \land \lambda_{aux6=false} \Leftrightarrow \theta_{aux6=false|friends(a,b)=false \land smokes(b)=false}\\
\lambda_{friends(a,c)=true} \land \lambda_{smokes(c)=true} \land \lambda_{aux7=true} \Leftrightarrow \theta_{aux7=true|friends(a,c)=true \land smokes(c)=true}\\
\lambda_{friends(a,c)=true} \land \lambda_{smokes(c)=true} \land \lambda_{aux7=false} \Leftrightarrow \theta_{aux7=false|friends(a,c)=true \land smokes(c)=true}\\
\lambda_{friends(a,c)=true} \land \lambda_{smokes(c)=false} \land \lambda_{aux7=true} \Leftrightarrow \theta_{aux7=true|friends(a,c)=true \land smokes(c)=false}\\
\lambda_{friends(a,c)=true} \land \lambda_{smokes(c)=false} \land \lambda_{aux7=false} \Leftrightarrow \theta_{aux7=false|friends(a,c)=true \land smokes(c)=false}\\
\lambda_{friends(a,c)=false} \land \lambda_{smokes(c)=true} \land \lambda_{aux7=true} \Leftrightarrow \theta_{aux7=true|friends(a,c)=false \land smokes(c)=true}\\
\lambda_{friends(a,c)=false} \land \lambda_{smokes(c)=true} \land \lambda_{aux7=false} \Leftrightarrow \theta_{aux7=false|friends(a,c)=false \land smokes(c)=true}\\
\lambda_{friends(a,c)=false} \land \lambda_{smokes(c)=false} \land \lambda_{aux7=true} \Leftrightarrow \theta_{aux7=true|friends(a,c)=false \land smokes(c)=false}\\
\lambda_{friends(a,c)=false} \land \lambda_{smokes(c)=false} \land \lambda_{aux7=false} \Leftrightarrow \theta_{aux7=false|friends(a,c)=false \land smokes(c)=false}\\
\lambda_{aux1=true} \land \lambda_{aux6=true} \land \lambda_{aux7=true} \land \lambda_{smokes(a)=true} \Leftrightarrow \theta_{smokes(a)=true|aux1=true \land aux6=true \land aux7=true}\\
\lambda_{aux1=true} \land \lambda_{aux6=true} \land \lambda_{aux7=true} \land \lambda_{smokes(a)=false} \Leftrightarrow \theta_{smokes(a)=false|aux1=true \land aux6=true \land aux7=true}\\
\lambda_{aux1=true} \land \lambda_{aux6=true} \land \lambda_{aux7=false} \land \lambda_{smokes(a)=true} \Leftrightarrow \theta_{smokes(a)=true|aux1=true \land aux6=true \land aux7=false}\\
\lambda_{aux1=true} \land \lambda_{aux6=true} \land \lambda_{aux7=false} \land \lambda_{smokes(a)=false} \Leftrightarrow \theta_{smokes(a)=false|aux1=true \land aux6=true \land aux7=false}\\
\lambda_{aux1=true} \land \lambda_{aux6=false} \land \lambda_{aux7=true} \land \lambda_{smokes(a)=true} \Leftrightarrow \theta_{smokes(a)=true|aux1=true \land aux6=false \land aux7=true}\\
\lambda_{aux1=true} \land \lambda_{aux6=false} \land \lambda_{aux7=true} \land \lambda_{smokes(a)=false} \Leftrightarrow \theta_{smokes(a)=false|aux1=true \land aux6=false \land aux7=true}\\
\lambda_{aux1=true} \land \lambda_{aux6=false} \land \lambda_{aux7=false} \land \lambda_{smokes(a)=true} \Leftrightarrow \theta_{smokes(a)=true|aux1=true \land aux6=false \land aux7=false}\\
\lambda_{aux1=true} \land \lambda_{aux6=false} \land \lambda_{aux7=false} \land \lambda_{smokes(a)=false} \Leftrightarrow \theta_{smokes(a)=false|aux1=true \land aux6=false \land aux7=false}\\
\lambda_{aux1=false} \land \lambda_{aux6=true} \land \lambda_{aux7=true} \land \lambda_{smokes(a)=true} \Leftrightarrow \theta_{smokes(a)=true|aux1=false \land aux6=true \land aux7=true}\\
\lambda_{aux1=false} \land \lambda_{aux6=true} \land \lambda_{aux7=true} \land \lambda_{smokes(a)=false} \Leftrightarrow \theta_{smokes(a)=false|aux1=false \land aux6=true \land aux7=true}\\
\lambda_{aux1=false} \land \lambda_{aux6=true} \land \lambda_{aux7=false} \land \lambda_{smokes(a)=true} \Leftrightarrow \theta_{smokes(a)=true|aux1=false \land aux6=true \land aux7=false}\\
\lambda_{aux1=false} \land \lambda_{aux6=true} \land \lambda_{aux7=false} \land \lambda_{smokes(a)=false} \Leftrightarrow \theta_{smokes(a)=false|aux1=false \land aux6=true \land aux7=false}\\
\lambda_{aux1=false} \land \lambda_{aux6=false} \land \lambda_{aux7=true} \land \lambda_{smokes(a)=true} \Leftrightarrow \theta_{smokes(a)=true|aux1=false \land aux6=false \land aux7=true}\\
\lambda_{aux1=false} \land \lambda_{aux6=false} \land \lambda_{aux7=true} \land \lambda_{smokes(a)=false} \Leftrightarrow \theta_{smokes(a)=false|aux1=false \land aux6=false \land aux7=true}\\
\lambda_{aux1=false} \land \lambda_{aux6=false} \land \lambda_{aux7=false} \land \lambda_{smokes(a)=true} \Leftrightarrow \theta_{smokes(a)=true|aux1=false \land aux6=false \land aux7=false}\\
\lambda_{aux1=false} \land \lambda_{aux6=false} \land \lambda_{aux7=false} \land \lambda_{smokes(a)=false} \Leftrightarrow \theta_{smokes(a)=false|aux1=false \land aux6=false \land aux7=false}
$

\fakesubsubsection{ENC2 (Parameter Clauses)}

$
\rho_{stress(a)=true} \Rightarrow \lambda_{stress(a)=true}\\
\lnot\rho_{stress(a)=true} \Rightarrow \lambda_{stress(a)=false}\\
\rho_{stress(b)=true} \Rightarrow \lambda_{stress(b)=true}\\
\lnot\rho_{stress(b)=true} \Rightarrow \lambda_{stress(b)=false}\\
\rho_{stress(c)=true} \Rightarrow \lambda_{stress(c)=true}\\
\lnot\rho_{stress(c)=true} \Rightarrow \lambda_{stress(c)=false}\\
\lambda_{stress(a)=true} \land \rho_{aux1=true} \Rightarrow \lambda_{aux1=true}\\
\lambda_{stress(a)=true} \land \lnot\rho_{aux1=true} \Rightarrow \lambda_{aux1=false}\\
\lambda_{stress(a)=false} \land \rho_{aux1=true} \Rightarrow \lambda_{aux1=true}\\
\lambda_{stress(a)=false} \land \lnot\rho_{aux1=true} \Rightarrow \lambda_{aux1=false}\\
\lambda_{stress(b)=true} \land \rho_{aux2=true} \Rightarrow \lambda_{aux2=true}\\
\lambda_{stress(b)=true} \land \lnot\rho_{aux2=true} \Rightarrow \lambda_{aux2=false}\\
\lambda_{stress(b)=false} \land \rho_{aux2=true} \Rightarrow \lambda_{aux2=true}\\
\lambda_{stress(b)=false} \land \lnot\rho_{aux2=true} \Rightarrow \lambda_{aux2=false}\\
\lambda_{stress(c)=true} \land \rho_{aux3=true} \Rightarrow \lambda_{aux3=true}\\
\lambda_{stress(c)=true} \land \lnot\rho_{aux3=true} \Rightarrow \lambda_{aux3=false}\\
\lambda_{stress(c)=false} \land \rho_{aux3=true} \Rightarrow \lambda_{aux3=true}\\
\lambda_{stress(c)=false} \land \lnot\rho_{aux3=true} \Rightarrow \lambda_{aux3=false}\\
\rho_{friends(a,b)=true} \Rightarrow \lambda_{friends(a,b)=true}\\
\lnot\rho_{friends(a,b)=true} \Rightarrow \lambda_{friends(a,b)=false}\\
\rho_{friends(a,c)=true} \Rightarrow \lambda_{friends(a,c)=true}\\
\lnot\rho_{friends(a,c)=true} \Rightarrow \lambda_{friends(a,c)=false}\\
\rho_{friends(b,c)=true} \Rightarrow \lambda_{friends(b,c)=true}\\
\lnot\rho_{friends(b,c)=true} \Rightarrow \lambda_{friends(b,c)=false}\\
\rho_{friends(c,b)=true} \Rightarrow \lambda_{friends(c,b)=true}\\
\lnot\rho_{friends(c,b)=true} \Rightarrow \lambda_{friends(c,b)=false}\\
\lambda_{friends(b,c)=true} \land \lambda_{aux3=true} \land \rho_{aux4=true} \Rightarrow \lambda_{aux4=true}\\
\lambda_{friends(b,c)=true} \land \lambda_{aux3=true} \land \lnot\rho_{aux4=true} \Rightarrow \lambda_{aux4=false}\\
\lambda_{friends(b,c)=true} \land \lambda_{aux3=false} \land \rho_{aux4=true} \Rightarrow \lambda_{aux4=true}\\
\lambda_{friends(b,c)=true} \land \lambda_{aux3=false} \land \lnot\rho_{aux4=true} \Rightarrow \lambda_{aux4=false}\\
\lambda_{friends(b,c)=false} \land \lambda_{aux3=true} \land \rho_{aux4=true} \Rightarrow \lambda_{aux4=true}\\
\lambda_{friends(b,c)=false} \land \lambda_{aux3=true} \land \lnot\rho_{aux4=true} \Rightarrow \lambda_{aux4=false}\\
\lambda_{friends(b,c)=false} \land \lambda_{aux3=false} \land \rho_{aux4=true} \Rightarrow \lambda_{aux4=true}\\
\lambda_{friends(b,c)=false} \land \lambda_{aux3=false} \land \lnot\rho_{aux4=true} \Rightarrow \lambda_{aux4=false}\\
\lambda_{friends(c,b)=true} \land \lambda_{aux2=true} \land \rho_{aux5=true} \Rightarrow \lambda_{aux5=true}\\
\lambda_{friends(c,b)=true} \land \lambda_{aux2=true} \land \lnot\rho_{aux5=true} \Rightarrow \lambda_{aux5=false}\\
\lambda_{friends(c,b)=true} \land \lambda_{aux2=false} \land \rho_{aux5=true} \Rightarrow \lambda_{aux5=true}\\
\lambda_{friends(c,b)=true} \land \lambda_{aux2=false} \land \lnot\rho_{aux5=true} \Rightarrow \lambda_{aux5=false}\\
\lambda_{friends(c,b)=false} \land \lambda_{aux2=true} \land \rho_{aux5=true} \Rightarrow \lambda_{aux5=true}\\
\lambda_{friends(c,b)=false} \land \lambda_{aux2=true} \land \lnot\rho_{aux5=true} \Rightarrow \lambda_{aux5=false}\\
\lambda_{friends(c,b)=false} \land \lambda_{aux2=false} \land \rho_{aux5=true} \Rightarrow \lambda_{aux5=true}\\
\lambda_{friends(c,b)=false} \land \lambda_{aux2=false} \land \lnot\rho_{aux5=true} \Rightarrow \lambda_{aux5=false}\\
\lambda_{aux4=true} \land \lambda_{aux2=true} \land \rho_{smokes(b)=true} \Rightarrow \lambda_{smokes(b)=true}\\
\lambda_{aux4=true} \land \lambda_{aux2=true} \land \lnot\rho_{smokes(b)=true} \Rightarrow \lambda_{smokes(b)=false}\\
\lambda_{aux4=true} \land \lambda_{aux2=false} \land \rho_{smokes(b)=true} \Rightarrow \lambda_{smokes(b)=true}\\
\lambda_{aux4=true} \land \lambda_{aux2=false} \land \lnot\rho_{smokes(b)=true} \Rightarrow \lambda_{smokes(b)=false}\\
\lambda_{aux4=false} \land \lambda_{aux2=true} \land \rho_{smokes(b)=true} \Rightarrow \lambda_{smokes(b)=true}\\
\lambda_{aux4=false} \land \lambda_{aux2=true} \land \lnot\rho_{smokes(b)=true} \Rightarrow \lambda_{smokes(b)=false}\\
\lambda_{aux4=false} \land \lambda_{aux2=false} \land \rho_{smokes(b)=true} \Rightarrow \lambda_{smokes(b)=true}\\
\lambda_{aux4=false} \land \lambda_{aux2=false} \land \lnot\rho_{smokes(b)=true} \Rightarrow \lambda_{smokes(b)=false}\\
\lambda_{aux5=true} \land \lambda_{aux3=true} \land \rho_{smokes(c)=true} \Rightarrow \lambda_{smokes(c)=true}\\
\lambda_{aux5=true} \land \lambda_{aux3=true} \land \lnot\rho_{smokes(c)=true} \Rightarrow \lambda_{smokes(c)=false}\\
\lambda_{aux5=true} \land \lambda_{aux3=false} \land \rho_{smokes(c)=true} \Rightarrow \lambda_{smokes(c)=true}\\
\lambda_{aux5=true} \land \lambda_{aux3=false} \land \lnot\rho_{smokes(c)=true} \Rightarrow \lambda_{smokes(c)=false}\\
\lambda_{aux5=false} \land \lambda_{aux3=true} \land \rho_{smokes(c)=true} \Rightarrow \lambda_{smokes(c)=true}\\
\lambda_{aux5=false} \land \lambda_{aux3=true} \land \lnot\rho_{smokes(c)=true} \Rightarrow \lambda_{smokes(c)=false}\\
\lambda_{aux5=false} \land \lambda_{aux3=false} \land \rho_{smokes(c)=true} \Rightarrow \lambda_{smokes(c)=true}\\
\lambda_{aux5=false} \land \lambda_{aux3=false} \land \lnot\rho_{smokes(c)=true} \Rightarrow \lambda_{smokes(c)=false}\\
\lambda_{friends(a,b)=true} \land \lambda_{smokes(b)=true} \land \rho_{aux6=true} \Rightarrow \lambda_{aux6=true}\\
\lambda_{friends(a,b)=true} \land \lambda_{smokes(b)=true} \land \lnot\rho_{aux6=true} \Rightarrow \lambda_{aux6=false}\\
\lambda_{friends(a,b)=true} \land \lambda_{smokes(b)=false} \land \rho_{aux6=true} \Rightarrow \lambda_{aux6=true}\\
\lambda_{friends(a,b)=true} \land \lambda_{smokes(b)=false} \land \lnot\rho_{aux6=true} \Rightarrow \lambda_{aux6=false}\\
\lambda_{friends(a,b)=false} \land \lambda_{smokes(b)=true} \land \rho_{aux6=true} \Rightarrow \lambda_{aux6=true}\\
\lambda_{friends(a,b)=false} \land \lambda_{smokes(b)=true} \land \lnot\rho_{aux6=true} \Rightarrow \lambda_{aux6=false}\\
\lambda_{friends(a,b)=false} \land \lambda_{smokes(b)=false} \land \rho_{aux6=true} \Rightarrow \lambda_{aux6=true}\\
\lambda_{friends(a,b)=false} \land \lambda_{smokes(b)=false} \land \lnot\rho_{aux6=true} \Rightarrow \lambda_{aux6=false}\\
\lambda_{friends(a,c)=true} \land \lambda_{smokes(c)=true} \land \rho_{aux7=true} \Rightarrow \lambda_{aux7=true}\\
\lambda_{friends(a,c)=true} \land \lambda_{smokes(c)=true} \land \lnot\rho_{aux7=true} \Rightarrow \lambda_{aux7=false}\\
\lambda_{friends(a,c)=true} \land \lambda_{smokes(c)=false} \land \rho_{aux7=true} \Rightarrow \lambda_{aux7=true}\\
\lambda_{friends(a,c)=true} \land \lambda_{smokes(c)=false} \land \lnot\rho_{aux7=true} \Rightarrow \lambda_{aux7=false}\\
\lambda_{friends(a,c)=false} \land \lambda_{smokes(c)=true} \land \rho_{aux7=true} \Rightarrow \lambda_{aux7=true}\\
\lambda_{friends(a,c)=false} \land \lambda_{smokes(c)=true} \land \lnot\rho_{aux7=true} \Rightarrow \lambda_{aux7=false}\\
\lambda_{friends(a,c)=false} \land \lambda_{smokes(c)=false} \land \rho_{aux7=true} \Rightarrow \lambda_{aux7=true}\\
\lambda_{friends(a,c)=false} \land \lambda_{smokes(c)=false} \land \lnot\rho_{aux7=true} \Rightarrow \lambda_{aux7=false}\\
\lambda_{aux1=true} \land \lambda_{aux6=true} \land \lambda_{aux7=true} \land \rho_{smokes(a)=true} \Rightarrow \lambda_{smokes(a)=true}\\
\lambda_{aux1=true} \land \lambda_{aux6=true} \land \lambda_{aux7=true} \land \lnot\rho_{smokes(a)=true} \Rightarrow \lambda_{smokes(a)=false}\\
\lambda_{aux1=true} \land \lambda_{aux6=true} \land \lambda_{aux7=false} \land \rho_{smokes(a)=true} \Rightarrow \lambda_{smokes(a)=true}\\
\lambda_{aux1=true} \land \lambda_{aux6=true} \land \lambda_{aux7=false} \land \lnot\rho_{smokes(a)=true} \Rightarrow \lambda_{smokes(a)=false}\\
\lambda_{aux1=true} \land \lambda_{aux6=false} \land \lambda_{aux7=true} \land \rho_{smokes(a)=true} \Rightarrow \lambda_{smokes(a)=true}\\
\lambda_{aux1=true} \land \lambda_{aux6=false} \land \lambda_{aux7=true} \land \lnot\rho_{smokes(a)=true} \Rightarrow \lambda_{smokes(a)=false}\\
\lambda_{aux1=true} \land \lambda_{aux6=false} \land \lambda_{aux7=false} \land \rho_{smokes(a)=true} \Rightarrow \lambda_{smokes(a)=true}\\
\lambda_{aux1=true} \land \lambda_{aux6=false} \land \lambda_{aux7=false} \land \lnot\rho_{smokes(a)=true} \Rightarrow \lambda_{smokes(a)=false}\\
\lambda_{aux1=false} \land \lambda_{aux6=true} \land \lambda_{aux7=true} \land \rho_{smokes(a)=true} \Rightarrow \lambda_{smokes(a)=true}\\
\lambda_{aux1=false} \land \lambda_{aux6=true} \land \lambda_{aux7=true} \land \lnot\rho_{smokes(a)=true} \Rightarrow \lambda_{smokes(a)=false}\\
\lambda_{aux1=false} \land \lambda_{aux6=true} \land \lambda_{aux7=false} \land \rho_{smokes(a)=true} \Rightarrow \lambda_{smokes(a)=true}\\
\lambda_{aux1=false} \land \lambda_{aux6=true} \land \lambda_{aux7=false} \land \lnot\rho_{smokes(a)=true} \Rightarrow \lambda_{smokes(a)=false}\\
\lambda_{aux1=false} \land \lambda_{aux6=false} \land \lambda_{aux7=true} \land \rho_{smokes(a)=true} \Rightarrow \lambda_{smokes(a)=true}\\
\lambda_{aux1=false} \land \lambda_{aux6=false} \land \lambda_{aux7=true} \land \lnot\rho_{smokes(a)=true} \Rightarrow \lambda_{smokes(a)=false}\\
\lambda_{aux1=false} \land \lambda_{aux6=false} \land \lambda_{aux7=false} \land \rho_{smokes(a)=true} \Rightarrow \lambda_{smokes(a)=true}\\
\lambda_{aux1=false} \land \lambda_{aux6=false} \land \lambda_{aux7=false} \land \lnot\rho_{smokes(a)=true} \Rightarrow \lambda_{smokes(a)=false}
$


% 
% References
%

\bibliographystyle{plain}
\bibliography{references}
%\addcontentsline{toc}{section}{References}

\end{document}
